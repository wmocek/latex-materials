\documentclass[12pt,a4paper]{article}

\usepackage[polish]{babel}
\usepackage[T1]{fontenc}
\usepackage[utf8]{inputenc}
\usepackage{amsmath,amssymb}
\usepackage{geometry}
\geometry{margin=2.5cm}

\begin{document}

\begin{center}
\textbf{Próbny egzamin maturalny z Nową Erą}\\
\textbf{Matematyka – poziom rozszerzony}
\end{center}

\bigskip

\noindent
\textbf{Zadanie 4. \;(0--3)}

\medskip

Ciąg $(a_n)$ jest określony dla każdej liczby naturalnej $n \geq 1$ wzorem
\[
a_n =
\left[ 2 + 4 + 6 + \dots + (2n + 4) \right]
- \binom{2n}{2},
\]
gdzie $2 + 4 + 6 + \dots + (2n + 4)$ jest sumą kolejnych liczb naturalnych parzystych.

\medskip

Wyznacz $k$, jeśli wiadomo, że
\[
\frac{a_{k+1}}{a_k} = 10.
\]
Zapisz obliczenia.

\end{document}