\documentclass{beamer}

% ----- Encoding & fonts (must come first) -----
\usepackage[utf8]{inputenc}
\usepackage[T1]{fontenc}
\usepackage{lmodern}
\usepackage{bookmark}

% ----- Language (polish) -----
\usepackage[polish]{babel}

% ----- Theme -----
\usetheme{Madrid}
\usecolortheme{default}

% ----- Notes -----
\usepackage{pgfpages}
\setbeameroption{show notes on second screen=right}

% ----- Math + TikZ + PGF -----
\usepackage{amsmath}
\usepackage{amssymb}
\usepackage{tikz}
\usetikzlibrary{arrows.meta, shapes.geometric, positioning, calc, decorations.pathreplacing}
\usepackage{pgfplots}
\pgfplotsset{compat=1.18}

% ----- Title -----
\title{Wartość bezwzględna}
\author{}
\date{}

\begin{document}

% ---------- TITLE SLIDE ----------
\begin{frame}
    \titlepage
    \note{
        Przywitaj się z grupą, zapytaj jak minął dzień. Przedstaw się, zrób to na wesoło. :-) 
        Zapytaj się czy wiedzą coś na temat wartości bezwzględnej.
        
    }
\end{frame}

% ---------- FRAME 1 ----------
\begin{frame}{Wartość bezwzględna jako odległość od zera}
Interpretacja geometryczna:
\[
|x| = \text{odległość punktu } x \text{ od } 0 \text{ na osi liczbowej}.
\]

\begin{center}
\begin{tikzpicture}[scale=1.2, every node/.style={font=\normalsize}]
  % Number line with arrow
  \draw[-{Latex[length=3mm]}, line width=1.2pt] (-5.5,0) -- (5.5,0);
  
  % Tick marks and labels
  \foreach \x in {-5,-4,-3,-2,-1,1,2,3,4,5}{
    \draw[line width=0.8pt] (\x,0.1) -- (\x,-0.1) node[below=4pt] {$\x$};
  }
  
  % Zero point (emphasized)
  \draw[line width=0.8pt] (0,0.1) -- (0,-0.1);
  \filldraw[black] (0,0) circle (2.5pt) node[above=8pt] {$\mathbf{0}$};
  
  % Points x and -x
  \coordinate (P) at (3,0);
  \coordinate (N) at (-3,0);
  \filldraw[blue!70!black] (P) circle (3pt) node[above=8pt] {$\mathbf{x}$};
  \filldraw[red!70!black] (N) circle (3pt) node[above=8pt] {$\mathbf{-x}$};
  
  % Distance arrows with labels
  \draw[<->, line width=1.5pt, blue!70!black] (0,-0.8) -- (3,-0.8) 
    node[midway, below=4pt, fill=white, inner sep=2pt] {$|x|=3$};
  \draw[<->, line width=1.5pt, red!70!black] (-3,-0.8) -- (0,-0.8) 
    node[midway, below=4pt, fill=white, inner sep=2pt] {$|-x|=3$};
  
  % Decorative brace
  \draw[decorate, decoration={brace, amplitude=8pt, mirror}, line width=1pt]
    (0,1.2) -- (3,1.2) node[midway, above=10pt] {\textbf{odległość od 0}};
  \draw[decorate, decoration={brace, amplitude=8pt, mirror}, line width=1pt]
    (-3,1.2) -- (0,1.2) node[midway, above=10pt] {\textbf{odległość od 0}};
\end{tikzpicture}
\end{center}

Wartość bezwzględna liczby rzeczywistej $x$ to:
\[
|x| = 
\begin{cases}
x, & x \ge 0,\\
-x, & x < 0.
\end{cases}
\]

\note{
Powiedz tutaj o wartości bezwzględnej jako odległości od zera. Zadaj najpierw pytania o odległość liczby 5 oraz -5 od zera itd. Następnie pokaż jak to się przekłada na znak minus gdy liczba jest ujemna itd.
}

\end{frame}

% ---------- FRAME 2 ----------
\begin{frame}{Wartość bezwzględna jako odległość od punktu $a$}
Wyrażenie $|x - a|$ oznacza odległość punktu $x$ od punktu $a$.

\[
|x-a| = \text{odległość między } x \text{ i } a.
\]

\[
|x-a| = 
\begin{cases}
x-a, & x \ge a,\\
a-x, & x < a.
\end{cases}
\]

\note{
Powiedz tutaj o wartości bezwzględnej jako odległości od od liczby $|x-a|$. Wytłumacz jak to działa.
}
\end{frame}

% ---------- FRAME 3 ----------
\begin{frame}{Proste równania i nierówności z wartością bezwzględną}
\textbf{Przykład równania:}
\[
|x-3| = 5
\]
%Daje dwa rozwiązania:
%\[
%x - 3 = 5 \quad \text{lub} \quad x - 3 = -5
%\]
%\[
%x = 8 \quad \text{lub} \quad x = -2
%\]

\textbf{Przykłady nierówności:}

$$|x-2| < 4$$
oraz:
$$|x+1| > 5$$
%co oznacza:
%\[
%-4 < x-2 < 4
%\]
%\[
%-2 < x < 6
%\]
\end{frame}

% ---------- FRAME 4 ----------
\begin{frame}{Trudniejsze przykłady}
\[
|x-1| + |x+2| = 7
\]

Rozwiązujemy, rozpatrując przedziały wyznaczone przez punkty $-2$ i $1$.

\bigskip

\[
|2x - 3| \ge |x + 1|
\]

Podnosimy obie strony do kwadratu lub rozważamy przypadki.
\end{frame}

% ---------- FRAME 5 ----------
\begin{frame}{Zadanie tekstowe — wartość bezwzględna}
\textbf{Treść:}

Szosa z Zakopanego do Gdańska prowadzi przez Kraków i Warszawę. Odległość
z Krakowa do Warszawy wynosi 300 km. Pan Kowalski wyjechał samochodem
ze swojego domu w Krakowie, pojechał do swego przyjaciela, który mieszka
gdzieś przy drodze z Zakopanego do Gdańska, a następnie pojechał do War-
szawy. Łącznie przejechał 380 km. Gdzie mieszka przyjaciel pana Kowalskiego?


\note{
Odległość z Krakowa do miejsca zamieszkania przyjaciela to $|x|$. Odległość z miejsca zamieszkania przyjaciela do Warszawy to $|300 - x|$. Suma tych odległości to 380 km. Rozwiążemy to równanie.
}
\end{frame}

% ---------- FRAME 6 ----------
\begin{frame}{Rozwiązanie zadania tekstowego}
\[
|x| + |300 - x| = 380
\]
Jedynym rozwiązaniem jest:
\[
x = -40 \quad \text{lub} \quad x = 340.
\]

Możliwe miejsca:
\begin{itemize}
\item 40 km od Krakowa w stronę Zakopanego,
\item 340 km od Krakowa w stronę Gdańska.
\end{itemize}
\end{frame}

% ---------- FRAME 7 ----------
\begin{frame}{Figura $|x| + |y| = 1$ a okrąg}
\[
|x| + |y| = 1
\]
opisuje romb o wierzchołkach:
\[
(1,0), (-1,0), (0,1), (0,-1)
\]

\begin{center}
\begin{tikzpicture}[scale=1.0]
  \draw[->] (-1.4,0) -- (1.4,0) node[right] {$x$};
  \draw[->] (0,-1.4) -- (0,1.4) node[above] {$y$};
  \draw[blue, thick] (0,0) circle (1cm);
  \draw[red, thick, rounded corners=1pt] (1,0) -- (0,1) -- (-1,0) -- (0,-1) -- cycle;
  \foreach \p in {(1,0),(0,1),(-1,0),(0,-1)}
    \filldraw \p circle (1.2pt);
\end{tikzpicture}
\end{center}
\end{frame}

% ---------- FRAME 8 ----------
\begin{frame}
Równanie $|x| + |y| = 1$ to okrąg w \textbf{metryce taksówkowej}.
\end{frame}

\end{document}