\documentclass[12pt,a4paper]{article}

\usepackage[utf8]{inputenc}
\usepackage[T1]{fontenc}
\usepackage[english]{babel}
\usepackage{hyperref}

\title{LaTeX Document with Bibliography}
\author{Your Name}
\date{\today}

\begin{document}

\maketitle

\section{Introduction}

This document demonstrates how to use citations and create a bibliography in LaTeX. According to \cite{knuth1984texbook}, LaTeX is a powerful typesetting system.

\section{Background}

The development of computer science has been well documented \cite{turing1936computable}. Modern research continues to build on these foundations \cite{dijkstra1968goto}.

\section{Methodology}

Our approach follows established practices in the field \cite{knuth1984texbook, dijkstra1968goto}.

\section{Conclusion}

This example shows how to integrate citations naturally into your text. See \cite{turing1936computable} for more details.

\begin{thebibliography}{9}

\bibitem{knuth1984texbook}
Donald E. Knuth.
\textit{The TeXbook}.
Addison-Wesley Professional, 1984.

\bibitem{turing1936computable}
Alan M. Turing.
On computable numbers, with an application to the Entscheidungsproblem.
\textit{Proceedings of the London Mathematical Society}, 42(2):230--265, 1936.

\bibitem{dijkstra1968goto}
Edsger W. Dijkstra.
Go to statement considered harmful.
\textit{Communications of the ACM}, 11(3):147--148, 1968.

\end{thebibliography}

\end{document}
